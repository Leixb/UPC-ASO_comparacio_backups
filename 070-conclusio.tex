%! TEX root = **/010-main.tex
% vim: spell spelllang=ca:

\section{Conclusions}%
\label{sec:conclusions}

\newcommand{\tick}{\ding{51}}

\begin{table}[H]
    \centering
    \caption{Comparativa
        \cite{noauthor_bacula_2014,noauthor_comparison_2020,noauthor_list_2020}
}
    \begin{tabular}{@{}lccccc@{}}
        \toprule
        & \multicolumn{2}{c}{Veeam} & \multicolumn{2}{c}{Bacula} &
        \multirow{2}{*}{BackupPC} \\
        & Community & Enterprise+ & Community & Enterprise & \\
        \midrule
        %                                 & V com & V ent & B com & B ent & Back &
        Codi lliure                       &       &       & \tick &       & \tick \\
        Compressió                        & \tick & \tick & \tick & \tick & \tick \\
        Encriptat                         & \tick & \tick & \tick & \tick &       \\
        Deduplicació                      &       & \tick &       & \tick &       \\
        Màquines Virtuals                 & \tick & \tick & \tick & \tick &       \\
        Snapshots                         & \tick & \tick &       & \tick &       \\
        Bases de dades                    & \tick & \tick & \tick & \tick &       \\
        Integració al núvol \footnotemark & \tick & \tick &       & \tick &       \\
        Copia en cinta                    & \tick & \tick & \tick & \tick &       \\
        Preu                              & FREE  & a negociar & FREE  &  a negociar    & FREE  \\
        Assistència tècnica               &       & \tick &       & \tick &       \\
        \bottomrule
    \end{tabular}
\end{table}

\footnotetext{AWS, Azure, \dots}

A la taula podem veure un resum de les diferencies entre els diferents programes
estudiats en el document. Podem veure que \emph{BackupPC} no te gaires
funcionalitats en comparació a la resta, però cal recordar que té un cas d'ús
diferent a \emph{Bacula} o \emph{Veeam}. Pel que fa a les edicions
\emph{community}, \emph{Bacula} ofereix menys funcionalitats que \emph{Veeam}.
Les edicions d'empresa en general implementen les mateixes funcionalitats, on es
diferencies son en les integracions amb altres productes (\emph{Veeam} ofereix
molta més varietat) i la complexitat de configuració. \emph{Bacula} és més
complex de configurar però permet ajustar millor el programa a les necessitats
de l'usuari. Un altre factor a considerar es tot l'ecosistema de \emph{Veeam}
que a part de les solucions estricament de copies de seguretat ofereix moltes
eines per inspeccionar les copies i realitzar altres tasques especialitzades que
\emph{Vacula} no ofereix.

Així doncs, el producte més complet és \emph{Veeam}. \emph{Bacula}, malgrat la
seva complexitat de configuració té més compatibilitat per fer copies de
seguretat en hardware local. Finalment, \emph{BackupPC} es una eina simple per a
fer copies de seguretat en disc a un sistema centralitzat amb configuració
mínima. Depenent del cas d'ús de cada empresa poden ser opcions viables.
