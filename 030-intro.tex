%! TEX root = **/010-main.tex
% vim: spell spelllang=ca:

%%%%%%%%%%%%%%%%%%%%%%%%%%%%%%%%%%%%%%%%%%%%%%%%%%%%%%%%%%%%%%%%%%%%%%%%%%%%%%%%
% ENUNCIAT
%%%%%%%%%%%%%%%%%%%%%%%%%%%%%%%%%%%%%%%%%%%%%%%%%%%%%%%%%%%%%%%%%%%%%%%%%%%%%%%%

 % Es demana fer una comparativa entre diferents sistemes de backup. N'heu de
 % triar tres diferents i comparar-los (compte amb el copy paste). Aquesta
 % comparativa ha de contemplar coses bones i dolentes de cadascun d'ells. US
 % poso algun exemple de solucions de backup:

 % - Veeam
 % - Bacula
 % - El tercer o si voleu alguns diferents vosaltres mateixos.

 % Amb ~8 pàgines ja hauria de valer. No us hauria de portar més de 2 hores

\section{Introducció}%
\label{sec:intro}

El creixement digital de les empreses fa que cada cop sigui mes important
millorar i automatitzar els sistemes de protecció de dades. Actualment hi ha
molts productes disponibles per administrar copies de seguretat amb diferents
avantatges i inconvenients.

En aquest document s'analitzen els avantatges i inconvenients de tres de les
solucions més utilitzades per la gestió de copies de seguretat:

\begin{itemize}
    \item Veeam
    \item Bacula
    \item BackupPC
\end{itemize}

% Els punts principals que es valoraran seran:

% \begin{itemize}
%     \item Característiques (\textenglish{features})
%     \item Facilitat d'us / Corba d'aprenentatge
%     \item Compatibilitat
%     \item Documentació / Comunitat / Assistència tècnica
%     \item Desenvolupament actiu
%     \item Cost
% \end{itemize}
