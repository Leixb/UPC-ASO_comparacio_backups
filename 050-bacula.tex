%! TEX root = **/010-main.tex
% vim: spell spelllang=ca:

\section{Bacula}%
\label{sec:bacula}

\begin{figure}[H]
    \centering
    \includegraphics[width=0.5\textwidth]{Bacula_logo}
\end{figure}


\emph{Bacula} es un sistema de copies de seguretat basat en un model de
servidor/client en xarxa. Hi ha dues versions, la \emph{community} que és
programari lliure sota la llicencia GPL2 i la \emph{enterprise} que conté
codi propietari.

La versió \emph{enterprise} ofereix assistència tècnica, millor documentació,
més integracions i \emph{features}. Per exemple, la versió \emph{community}
no té administració de \emph{snapshot}, copies de seguretat iniciades des del
client, compressió en línia de comunicació o deduplicació entre altres
\cite{noauthor_enterprise_nodate}.

La configuració de \emph{Bacula} és significativament més complexa que
\emph{Veeam} però permet configurar cada detall del sistema de copia i
restauració. Tot i que la versió \emph{enterprise} ofereix múltiples plugins per
integrar \emph{Bacula} amb altres serveis, no te tants com \emph{Veeam}.
\emph{Bacula} es centra en la copia i restauració de copies de seguretat, no té
altres funcionalitats.


\subsection{BareOS}%
\label{sub:bareos}

\begin{figure}[H]
    \centering
    \includegraphics[width=0.5\textwidth]{Bareos_logo}
\end{figure}

El descontentament amb els mantenidors de \emph{Bacula} va portar a la creació
d'una fork anomenada \emph{BareOS} en la que totes les característiques estan
disponibles a la versió gratuïta i la subscripció es per
assistència tècnica o accés a un repositori amb binaris precompilats amb les
ultimes versions i \emph{bugfixes} \cite{noauthor_what_nodate-1,vermaden_why_2018}

Tot i no tenir la mateixa funcionalitat que \emph{Bacula Enterprise}, és una
opció a considerar.
