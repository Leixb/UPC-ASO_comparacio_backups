%! TEX root = **/010-main.tex
% vim: spell spelllang=ca:

\section{BackupPC}%
\label{sec:backuppc}

\begin{figure}[H]
    \centering
    \includegraphics[width=0.5\textwidth]{BackupPC_logo}
\end{figure}

\emph{BackupPC} és un programari lliure (GPL-v3.0) que permet fer copies de seguretat
disc a disc. No es necessari un programa client ja que el propi servidor
accedeix als fitxers de cada màquina a traves de \emph{ssh/NFS/rsync/rsh} (o
\emph{Samba} en Windows) \cite{noauthor_about_nodate}.

El programa esta escrit en \emph{Perl} i com a conseqüència es relativament fàcil
d'adaptar i extendre la funcionalitat a casos d'us concret. Tot i no estar
escrit en codi compilat, la majoria d'operacions al fer copies de seguretat son d'I/O
utilitzant \emph{tar}, per tant no hi ha problemes de rendiment rellevants
\footnote{Excepte que la versió de \emph{Perl} del sistema ha d'estar compilada
amb \emph{largefiles} si es volen fer copies de fitxers de mes de 2 Gb
}.

El programa és molt més simple que \emph{Veeam} o \emph{Bacula} ja que només
pretén fer copies de seguretat a un servidor central. No te cap tipus de
integració amb serveis del núvol, copia de maquines virtuals, copia de bases de
dades o fitxers en ús ni copia en cinta \cite{noauthor_backuppc_nodate}.
