%! TEX root = **/010-main.tex
% vim: spell spelllang=ca:

\section{Veeam}%
\label{sec:veeam}

\begin{figure}[H]
    \centering
    \includegraphics[width=0.5\textwidth]{Veeam_logo}
\end{figure}

\emph{Veeam} ofereix una suite completa per administrar copies de seguretat
tant de màquines virtuals, com físiques i al núvol (AWS, Microsoft Azure,
\dots) \cite{noauthor_veeam_2020}. Tots els seus productes son propietaris. En
els últims anys s'ha posicionat com un dels líders de la industria.

El principal atractiu és la seva simplicitat de configuració degut a al gran
varietat de productes de tercers amb els que es pot integrar. El nombre de
productes i serveis que suporta augmenta amb cada versió ja que te un
desenvolupament molt actiu (tot i ser propietari).

Un altre dels seus atractius es la API que permet integrar serveis i productes
que no son directament compatibles oferint flexibilitat a les empreses per
adaptar el producte als seu entorn concret.

\emph{Veeam} ofereix diversos productes en funció de les necessitats de
l'empresa i una llicència universal. Això permet a cada empresa crear el seu
propi pressupost en funció de les seves necessitats. La llicencia universal
permet a les empreses pagar per llicencies de \emph{workloads} en comptes del
model tradicional d'altres empreses de pagar llicencies per màquines.
L'avantatge es que una llicencia es pot aplicar tant a màquines virtuals, com
físiques com a serveis del núvol. D'aquesta manera es pot fer una transició al
núvol sense haver de renovar les llicencies \cite{noauthor_veeam_nodate-1}.

Els seus productes per empreses ofereix assistència tècnica 24/7 arreu del món
amb sistema de tiquets i trucades. A part, també te un fòrum públic i una
extensa documentació sobre el producte i FAQs accessibles a tothom
\cite{noauthor_veeam_nodate-2}.

\emph{Veeam} és ideal per empreses que depenen de una gran varietat de serveis i
productes i busquen una solució segura i fàcil de configurar i mantenir. Els
seus principals inconvenients son el preu, el fet que el codi es propietari
i algunes limitacions dels productes.  Aquestes limitacions son en casos d'us
molt específics, però abans de comprar els productes es recomanable anar al
\emph{Help center} i buscar \emph{limitations} dins la secció del producte
\footnote{Exemple de limitacions d'integració amb sistemes d'emmagatzematge:
\url{https://helpcenter.veeam.com/docs/backup/vsphere/storage_limitations_general.html?ver=100}}.
